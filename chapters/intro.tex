\chapter{\ifenglish Introduction\else บทนำ\fi}

\section{\ifenglish Project rationale\else ที่มาของโครงงาน\fi}
โครงงานชิ้นนี้มีจุดเริ่มต้นจากปัญหาที่พบด้วยตัวเองในช่วงของการอ่านหนังสือสอบ เนื่องจากในปัจจุบันการหาพื้นที่สำหรับการอ่านหนังสือ หรือทำกิจกรรมต่าง ๆ นั้นเป็นปัญหาที่พบได้บ่อยในชีวิตประจำวัน ซึ่งอาจเกิดพบเจอปัญหาในเรื่องของการแย่งที่นั่ง หรือไม่มีสถานที่ให้บริการอย่างเพียงพอต่อการใช้งาน โดยเฉพาะในสถานที่สาธารณะ เช่น ห้องสมุด ห้องอ่านหนังสือ หรือพื้นที่ต่าง ๆ ที่มีการให้บริการสำหรับการอ่านหนังสือ และทำกิจกรรมต่าง ๆ อันมีสาเหตุหลักมาจากปริมาณผู้ใช้งานที่ต้องการใช้สถานที่นั้น ๆ ในเวลาใกล้เคียงกัน และในบางสถานที่ที่กล่าวมาข้างต้น อาจมีสิ่งอำนวยความสะดวกที่ไม่เพียงพอ หรือไม่ตอบโจทย์การใช้งาน เช่น จุดต่อปลั๊กไฟที่ไม่เพียงพอ และความสว่างของสถานที่ที่ให้บริการไม่เหมาะสม เป็นต้น

เพื่อแก้ไขปัญหาดังกล่าว โครงงานนี้จะมุ่งเน้นการสร้างเว็บแอปพลิเคชันสำหรับการจองที่นั่งในสถานที่สำหรับการอ่านหนังสือ และกิจกรรมต่าง ๆ ที่จะช่วยเพิ่มประสิทธิภาพในการใช้สถานที่ ลดปัญหาการแย่งที่นั่ง และเพิ่มความสะดวกสบายให้กับผู้ใช้งาน โดยผู้ใช้งานสามารถดูข้อมูลเกี่ยวกับสถานที่ที่ต้องการใช้บริการ จองที่นั่ง และเช็คสถานะการจองได้ผ่านเว็บแอปพลิเคชัน

\section{\ifenglish Objectives\else วัตถุประสงค์ของโครงงาน\fi}
\begin{enumerate}
    \item ลดปัญหาการแย่งที่นั่ง และเพิ่มประสิทธิภาพในการใช้สถานที่ โดยผู้ใช้งานสามารถจองที่นั่งล่วงหน้าได้ 
    \item เพิ่มความสะดวกสบายให้กับผู้ใช้งาน โดยผู้ใช้งานสามารถดูข้อมูลเกี่ยวกับสถานที่ที่ต้องการใช้บริการ ทำการจองที่นั่ง และเช็คสถานะการจองได้ผ่านเว็บแอปพลิเคชัน
    \item ผู้ให้บริการสถานที่สามารถวางแผนการใช้สถานที่ได้อย่างมีประสิทธิภาพ
    \item เพิ่มประสิทธิภาพในการจัดการสถานที่ โดยผู้ให้บริการสถานที่สามารถติดตามสถานะการจอง จัดการ booking และปรับปรุงการให้บริการได้อย่างสะดวกสบาย
\end{enumerate}

\section{\ifenglish Project scope\else ขอบเขตของโครงงาน\fi}

\subsection{\ifenglish Hardware scope\else ขอบเขตด้านฮาร์ดแวร์\fi}
\begin{enumerate}
    \item เป็นเว็บแอปพลิเคชันที่สามารถใช้งานได้โดยใช้ระบบการสัมผัสบนหน้าจอเพื่อควบคุมและปรับแต่งสิ่งต่างๆ
    \item เป็นเว็บแอปพลิเคชันที่แสดงผลผ่านทางหน้าจอ monitor และสามารถควบคุมการทำงานผ่าน mouse และ keyboard
\end{enumerate}

\subsection{\ifenglish Software scope\else ขอบเขตด้านซอฟต์แวร์\fi}
\begin{enumerate}
    \item เว็บแอปพลิเคชันที่พัฒนาด้วยหลักแนวคิดของ progressive web application (PWA) ซึ่งสามารถนำไปใช้ได้ในระบบปฎิบัติการต่าง ๆ ได้ทุกแพลตฟอร์ม เช่น iOS, iPadOS, MacOS, Android, Windows, Linux
\end{enumerate}

\section{\ifenglish Expected outcomes\else ประโยชน์ที่ได้รับ\fi}
\begin{enumerate}
    \item ผู้ใช้บริการจะมีพื้นที่สำหรับใช้ทำกิจกรรมต่าง ๆ ได้ตามความต้องการ เช่น อ่านหนังสือ, ทำงานกลุ่ม, ประชุม เป็นต้น
    \item ลดปัญหาในเรื่องของการแย่งที่นั่ง หรือไม่มีสถานที่ให้บริการอย่างเพียงพอต่อการใช้งาน
    \item ช่วยให้ผู้ประกอบการมีรายได้จากการปล่อยเช่าพื้นที่รายชั่วโมง
\end{enumerate}

\section{\ifenglish Technology and tools\else เทคโนโลยีและเครื่องมือที่ใช้\fi}

\subsection{\ifenglish Hardware technology\else เทคโนโลยีด้านฮาร์ดแวร์\fi}
\begin{enumerate}
    \item Macbook Pro 2019
    \item Desktop computer
    \item Laptop Asus ROG Zephyrus G14 2021
    \item iPad Pro รุ่น 11 นิ้ว (รุ่นที่2)
    \item iPhone SE (รุ่นที่3)
    \item Huawei P30 Pro
\end{enumerate}

\subsection{\ifenglish Software technology\else เทคโนโลยีด้านซอฟต์แวร์\fi}
\subsubsection{React TS~\cite{react_ts}}
React TS หรือ React with TypeScript คือการใช้ TypeScript ในการพัฒนาแอปพลิเคชันเว็บโดยใช้ React เป็นเฟรมเวิร์กหลัก โดย TypeScript เป็นภาษาโปรแกรมที่เพิ่มประสิทธิภาพให้กับ JavaScript โดยมีการเพิ่มความแม่นยำในการตรวจสอบประเภทของตัวแปรและการเข้าถึงข้อมูลต่าง ๆ ซึ่งช่วยลดข้อผิดพลาดที่เกิดจากการใช้งานตัวแปรและฟังก์ชันผิดประเภท โดย React TS สามารถช่วยให้การพัฒนาแอปพลิเคชันเว็บที่มีโค้ดมากขึ้นและซับซ้อนมีความน่าเชื่อถือและง่ายต่อการบำรุงรักษาได้ดีขึ้น นอกจากนี้ React TS ยังช่วยให้นักพัฒนาสามารถเขียนโค้ดได้เร็วขึ้น และช่วยลดความซับซ้อนของโค้ดและการแก้ไขข้อผิดพลาดในระยะยาว
\subsubsection{Sass~\cite{sass}}
Sass (Syntactically Awesome Style Sheets) เป็น preprocessor ของ CSS ที่ช่วยให้นักพัฒนาเว็บไซต์และออกแบบเว็บไซต์สามารถเขียน CSS ได้อย่างสะดวกสบายและมีประสิทธิภาพมากขึ้น โดย Sass ช่วยให้เราเขียน CSS ได้โดยไม่ต้องมีความซับซ้อน และช่วยลดการซ้ำซ้อนของโค้ด CSS ที่ซ้ำกันซ้ำข้าง ซึ่งช่วยให้การบริหารจัดการโค้ด CSS ของเว็บไซต์เป็นไปอย่างมีประสิทธิภาพมากขึ้น
\subsubsection{Vite~\cite{vite}}
Vite คือเครื่องมือสำหรับการพัฒนาเว็บแอปพลิเคชัน (web application) โดยเฉพาะสำหรับ Vue.js และ React.js ที่เป็นแพลตฟอร์มสำหรับการสร้างแอปพลิเคชันเว็บ (web apps) และเว็บไซต์ (websites) ที่มีประสิทธิภาพสูง โดยใช้โครงสร้างของ ES modules ในการโหลดโมดูลแทนการใช้ bundle แบบเดิมของ Webpack ทำให้การโหลดและแสดงผลหน้าเว็บไซต์ได้อย่างรวดเร็วและมีประสิทธิภาพสูงกว่า

Vite ช่วยให้นักพัฒนาสามารถสร้างและพัฒนาเว็บแอปพลิเคชันได้อย่างรวดเร็ว โดยใช้ development server ที่มี live reloading และ hot module replacement ซึ่งช่วยให้นักพัฒนาสามารถดูผลลัพธ์การเปลี่ยนแปลงของโค้ดได้ทันที นอกจากนี้ Vite ยังมี plugin ที่ช่วยให้นักพัฒนาสามารถใช้งาน CSS preprocessor เช่น Sass, Less และ Stylus ได้อย่างง่ายดาย และยังรองรับการใช้งาน TypeScript และ JSX สำหรับ React ได้อย่างเต็มรูปแบบ ทำให้ Vite เป็นเครื่องมือที่ช่วยให้นักพัฒนาสามารถพัฒนาเว็บแอปพลิเคชันได้อย่างรวดเร็วและมีประสิทธิภาพสูง
\subsubsection{Express.js~\cite{expressjs}}
Express.js เป็นเว็บเฟรมเวิร์ค (web framework) ของ Node.js ที่ช่วยให้นักพัฒนาสามารถสร้างแอปพลิเคชันเว็บได้อย่างรวดเร็วและง่ายขึ้น โดย Express.js จะช่วยในการจัดการเหตุการณ์และระบบเส้นทาง (routing) และการสร้างส่วนหลังบ้าน (backend) ของแอปพลิเคชันเว็บ นอกจากนี้ Express.js ยังรองรับการใช้งาน middleware ที่ช่วยในการจัดการข้อมูล การตรวจสอบการเข้าสู่ระบบ การเข้ารหัสลับและการส่งคำขอ HTTP ในรูปแบบต่างๆ เช่น JSON, XML, HTML, และอื่นๆ

Express.js เป็นเว็บเฟรมเวิร์คที่มีความยืดหยุ่นสูง และช่วยให้นักพัฒนาสามารถสร้างแอปพลิเคชันเว็บได้อย่างรวดเร็ว โดยไม่ต้องมีความรู้มากมายในการเขียนโค้ด Node.js หรือการจัดการเว็บเซิร์ฟเวอร์ (web server)
\subsubsection{Node.js~\cite{nodejs}}
Node.js เป็นแพลตฟอร์ม (platform) สำหรับการเขียนโปรแกรมภาษา JavaScript ซึ่งทำงานบนเครื่องเซิร์ฟเวอร์ (server) ซึ่งมีความเป็นที่นิยมในการพัฒนาแอปพลิเคชันเว็บ (web application) และแอปพลิเคชันโดยทั่วไป

Node.js มีการออกแบบมาเพื่อทำให้นักพัฒนาสามารถเขียนโค้ด JavaScript สำหรับเว็บไซต์และเครื่องเซิร์ฟเวอร์ได้อย่างมีประสิทธิภาพ โดยเฉพาะสำหรับการทำงานที่ต้องใช้ระบบ Input/Output อย่างเช่น การอ่านและเขียนไฟล์ การติดต่อกับฐานข้อมูล หรือการสื่อสารผ่านเครือข่าย (network communication) โดย Node.js สามารถทำงานร่วมกับหลายๆเทคโนโลยีที่ได้รับความนิยม เช่น Express.js, Socket.io, MongoDB, React.js และอื่นๆ ทำให้สามารถสร้างแอปพลิเคชันที่ทันสมัยและมีประสิทธิภาพสูงได้ง่ายขึ้น
\subsubsection{Stripe~\cite{stripe}}
Stripe เป็นบริการชำระเงินออนไลน์ (online payment) ที่ให้บริการในการรับชำระเงินออนไลน์ให้กับธุรกิจและองค์กรต่างๆ โดยมีความสามารถในการรับชำระเงินด้วยบัตรเครดิตและเดบิต รวมถึงการรับชำระเงินผ่านบริการอื่นๆ เช่น PromptPay QR(th), Apple Pay, Google Pay, และ Alipay

Stripe ได้รับความนิยมเป็นอย่างมากในวงการธุรกิจออนไลน์ เนื่องจากมีความปลอดภัยสูงในการรับชำระเงินออนไลน์ และมีส่วนของ API ที่ง่ายต่อการใช้งานและออกแบบมาอย่างดีเยี่ยม เพื่อให้นักพัฒนาสามารถนำมาปรับใช้กับแอปพลิเคชันต่างๆ ได้อย่างสะดวก นอกจากนี้ Stripe ยังมีความสามารถในการจัดการการคืนเงิน การสร้างและส่งใบเสร็จรับเงิน การจัดการลูกค้า และการดูแลการเงินขององค์กรได้อย่างครบวงจร
\subsubsection{PromptPay QR~\cite{promptpay}}
PromptPay QR เป็นรูปแบบการชำระเงินผ่านทางมือถือ ที่พัฒนาขึ้นโดยธนาคารแห่งประเทศไทย (ธปท.) เพื่อให้ผู้ใช้งานสามารถทำการชำระเงินผ่านแอปพลิเคชันของธนาคารหรือแอปพลิเคชันชำระเงินอื่นๆ ได้อย่างง่ายดาย

PromptPay QR มีการสร้างรหัส QR Code ที่เก็บข้อมูลของผู้รับเงิน ซึ่งสามารถสแกนด้วยแอปพลิเคชัน PromptPay หรือแอปพลิเคชันอื่นๆ ที่รองรับรูปแบบการชำระเงินด้วย PromptPay QR โดยข้อมูลที่จะถูกเก็บไว้ใน QR Code ประกอบด้วยหมายเลขโทรศัพท์หรือหมายเลขประจำตัวประชาชนของผู้รับเงิน หรือหมายเลขบัญชีธนาคารของผู้รับเงิน ทำให้ผู้ใช้งานสามารถทำการชำระเงินได้อย่างง่ายดายและปลอดภัย โดยสามารถใช้งานได้กับทุกธนาคารในประเทศไทยที่รองรับรูปแบบการชำระเงินด้วย PromptPay QR
\subsubsection{MySQL~\cite{mysql}}
MySQL คือระบบจัดการฐานข้อมูลแบบ Relational Database Management System (RDBMS) ซึ่งเป็นซอฟต์แวร์ที่ใช้สำหรับการจัดการฐานข้อมูล โดยมีความสามารถในการเก็บข้อมูลและการเรียกใช้งานข้อมูลได้อย่างมีประสิทธิภาพ สามารถใช้งานได้กับหลายภาษาโปรแกรมมิ่ง เช่น PHP, Python, Java เป็นต้น
\subsubsection{Sequelize.js~\cite{sequelize}}
Sequelize.js คือ ORM (Object-Relational Mapping) สำหรับ Node.js ที่ช่วยให้เราสามารถเข้าถึงฐานข้อมูลแบบ relational database ได้อย่างสะดวกและง่ายขึ้น โดยที่ไม่จำเป็นต้องเขียน SQL โดยตรง เราสามารถใช้ภาษา JavaScript ในการคิวรีฐานข้อมูลและจัดการข้อมูลได้

Sequelize.js มีฟีเจอร์ที่ช่วยให้การจัดการฐานข้อมูลเป็นเรื่องง่าย เช่น การสร้าง table, การเพิ่มและแก้ไขข้อมูลใน table, การคิวรีข้อมูลจาก table และอื่น ๆ อีกมากมาย นอกจากนี้ Sequelize.js ยังสามารถเชื่อมต่อกับหลายฐานข้อมูลได้ เช่น MySQL, PostgreSQL, SQLite, Microsoft SQL Server เป็นต้น
\subsubsection{Netlify~\cite{netlify}}
Netlify เป็นเว็บโฮสติ้งและบริการพัฒนาเว็บไซต์ที่ให้บริการฟรีหรือเสียค่าใช้จ่ายตามแพ็กเกจที่ใช้งาน โดย Netlify มีคุณสมบัติสำหรับสร้างและเชื่อมต่อเว็บไซต์จาก Git repository ให้กับ Netlify โดยอัตโนมัติ ทำให้เราสามารถทดสอบเว็บไซต์และส่งอัปเดตได้อย่างรวดเร็ว โดย Netlify ยังมีบริการต่าง ๆ เช่น CDN (Content Delivery Network) ที่ช่วยเพิ่มประสิทธิภาพในการโหลดเว็บไซต์ ระบบสำรองข้อมูลอัตโนมัติ และ SSL (Secure Sockets Layer) ที่ช่วยเพิ่มความปลอดภัยให้กับเว็บไซต์ของเรา
\subsubsection{Railway~\cite{railway}}
Railway คือบริการเว็บโฮสต์และจัดการแอปพลิเคชันที่ใช้สำหรับสร้างและใช้งาน APIs, backend และ databases ที่ช่วยให้ผู้พัฒนาสามารพัฒนาแอปพลิเคชัน Node.js ได้อย่างรวดเร็วและง่ายดาย โดยไม่จำเป็นต้องกังวลเรื่องการเซ็ตอัพและความซับซ้อนของการดูแลเซิร์ฟเวอร์ และฐานข้อมูล รวมถึงเรื่องการเชื่อมต่อกับบริการอื่น ๆ อีกด้วย ผู้ใช้สามารถเลือกใช้ฐานข้อมูลที่ต้องการเช่น PostgreSQL, MySQL, SQLite, MongoDB รวมถึงบริการอื่น ๆ เช่น Stripe และ Firebase ได้อย่างสะดวกสบาย และสามารถเปิด API สำหรับการเชื่อมต่อกับบริการอื่น ๆ ได้ง่ายดายด้วยเครื่องมือที่เตรียมไว้ให้ ทั้งหมดนี้ช่วยให้ผู้พัฒนาสามารถโฮสต์แอปพลิเคชันได้อย่างรวดเร็วและสะดวกสบายโดยไม่ต้องเสียเวลาในการกำหนดค่าและดูแลเซิร์ฟเวอร์และฐานข้อมูลของตนเองเอง
\subsubsection{Amazon RDS~\cite{amazon_rds}}
Amazon RDS (Relational Database Service) คือบริการฐานข้อมูลความสัมพันธ์ที่เป็นคลาวด์บน AWS (Amazon Web Services) ซึ่งมีฟีเจอร์การจัดการฐานข้อมูลอย่างเป็นมาตรฐานเหมาะสำหรับการพัฒนาและใช้งานแอปพลิเคชันขนาดใหญ่ที่มีการเข้าถึงข้อมูลจำนวนมาก ซึ่ง Amazon RDS ให้ความสามารถในการสร้าง, เปิดใช้งานและปรับขนาดฐานข้อมูลได้อย่างง่ายดาย และสามารถสนับสนุนการใช้งานฐานข้อมูล MySQL, PostgreSQL, Oracle, SQL Server, และ MariaDB ได้ โดยมีความมั่นคงสูง, ปลอดภัยและมีความสามารถในการดูแลรักษาข้อมูลเช่นการสำรองข้อมูลและเรียกคืนข้อมูลได้อย่างง่ายดายด้วยฟีเจอร์ Automated Backups และ Database Snapshots ที่มากับบริการนี้
\subsubsection{Firebase~\cite{firebase}}
Firebase คือบริการคลาวด์ที่พัฒนาโดย Google สำหรับการสร้างแอปพลิเคชันและเว็บไซต์ ซึ่งมีคุณสมบัติในการจัดการและบริการอย่างครบวงจรเกี่ยวกับแอปพลิเคชัน รวมถึงฐานข้อมูลแบบ NoSQL, โฮสต์เว็บไซต์และไฟล์, การตรวจสอบผู้ใช้, การจัดการการตั้งค่าแอปพลิเคชันและการแจ้งเตือน และอื่น ๆ โดย Firebase เป็นฟรีเท่ากับที่ให้บริการและมีแพลตฟอร์มที่สามารถใช้งานได้หลากหลายเช่น iOS, Android, JavaScript, Unity, C++, C\#, Python และ Node.js ทำให้นักพัฒนาสามารถพัฒนาแอปพลิเคชันได้อย่างรวดเร็ว และ Firebase ยังมีฟีเจอร์อื่น ๆ ที่ช่วยให้นักพัฒนาสามารถพัฒนาแอปพลิเคชันได้ง่ายและเร็วขึ้นเช่น Realtime Database, Cloud Firestore, Authentication, Cloud Functions, Cloud Messaging, Hosting, Crashlytics, Analytics และ ML Kit ฯลฯ
\subsubsection{Google Maps API~\cite{google_maps_api}}
Google Maps API เป็นชุดคำสั่งและบริการของ Google ที่ให้ผู้พัฒนาเข้าถึงข้อมูลแผนที่และบริการที่เกี่ยวข้องจาก Google Maps ได้ โดยนักพัฒนาสามารถใช้ Google Maps API เพื่อสร้างและแสดงแผนที่ในเว็บไซต์หรือแอปพลิเคชันของตนเองได้ ซึ่ง Google Maps API มีหลายรูปแบบเพื่อตอบสนองความต้องการของนักพัฒนา ได้แก่ Google Maps JavaScript API สำหรับใช้งานบนเว็บไซต์, Google Maps Android API สำหรับการพัฒนาแอปพลิเคชันบน Android, Google Maps iOS SDK สำหรับการพัฒนาแอปพลิเคชันบน iOS, และ Google Maps Places API สำหรับการเข้าถึงข้อมูลสถานที่และค้นหาสถานที่ที่ต้องการใน Google Maps ซึ่ง Google Maps API ยังมีฟีเจอร์อื่น ๆ อีกมากมายเช่นการเรียกใช้งานข้อมูลแผนที่ในรูปแบบระบบค้นหา, การแสดงผลตัวเลขสถิติการใช้งานแผนที่และการแสดงผลเส้นทางการเดินทาง ซึ่ง Google Maps API เป็นบริการที่มีค่าใช้จ่ายแต่มีแพ็กเกจฟรีสำหรับผู้ใช้งานที่มีการใช้งานรายได้ต่ำ หรือใช้งานเพื่อวัตถุประสงค์ที่ไม่เป็นธุรกิจ

\section{\ifenglish Project plan\else แผนการดำเนินงาน\fi}

\begin{plan}{10}{2022}{4}{2023}
    \planitem{10}{2022}{10}{2022}{ศึกษาค้นคว้าหาข้อมูลในการทำโครงงาน}
    \planitem{10}{2022}{12}{2022}{สรุป requirements ของ project}
    \planitem{11}{2022}{1}{2023}{ทำการออกแบบและแก้ไข UI/UX ของเว็บแอปพลิเคชัน}
    \planitem{12}{2022}{2}{2023}{ทำการออกแบบและแก้ไข schema ของ database}
    \planitem{12}{2022}{3}{2023}{พัฒนาเว็บแอปพลิเคชันในส่วนของ user interface}
    \planitem{12}{2022}{3}{2023}{พัฒนาเว็บแอปพลิเคชันในส่วนของ backend}
    \planitem{3}{2023}{3}{2023}{ติดต่อสอบถามข้อมูลกับทางผู้ให้บริการ (Flowider)}
    \planitem{3}{2023}{3}{2023}{ตรวจสอบ และทําการแก้ไขความเรียบร้อยของตัว project}
    \planitem{3}{2023}{3}{2023}{ทำการทดสอบฝั่งเว็บแอปพลิเคชัน Flowy \& Flowider}
    \planitem{3}{2023}{4}{2023}{เขียนรายงาน และส่ง project วิชา 261492}
    \caption{แผนการดําเนินงาน}
    \label{tab:plan}
\end{plan}

\section{\ifenglish Roles and responsibilities\else บทบาทและความรับผิดชอบ\fi}
\begin{itemize}
    \item นายชนาธิป สงจันทึก รับผิดชอบส่วนของ UX/UI design, front-end development, business relation
    \item นายอภิเทพ ปิยพิพัฒนมงคล รับผิดชอบในส่วนของ database schema, full stack development, business relation
\end{itemize}
\section{\ifenglish%
Impacts of this project on society, health, safety, legal, and cultural issues
\else%
ผลกระทบด้านสังคม สุขภาพ ความปลอดภัย กฎหมาย และวัฒนธรรม
\fi}

โครงงานนี้ผลกระทบในด้านของกฎหมายเป็นหลัก เพราะเป็นแพลตฟอร์มที่ให้บริการในส่วนของการจองพื้นที่นั่งสำหรับใช้ในการอ่านหนังสือ, ทํางานกลุ่ม, ประชุม โดยผู้ที่ใช้งานสามารถเลือกสถานที่ที่ผู้ใช้งานต้องการเข้าใช้บริการได้จากแพลตฟอร์ม Flowy และในฝั่งของผู็ให้บริการต้องทำการเพิ่มสถานที่เข้าไปในระบบด้วยแพลตฟอร์ม Flowider ซึ่งการลงทะเบียนเพื่อใช้บริการแพลตฟอร์มทั้งสองจำเป็นต้องเก็บรวบรวมข้อมูลส่วนบุคคลดังต่อนี้
\begin{itemize}
    \item ชื่อ - นามสกุล
    \item อีเมล
    \item เบอร์โทรศัพท์
    \item ชื่อธนาคาร \textit{(เฉพาะ Flowider)}
    \item หมายเลขบัญชีธนาคาร \textit{(เฉพาะ Flowider)}
    \item รหัสผ่าน
\end{itemize}
โดยทางผู้พัฒนาได้คำนึงถึงการเก็บรวบรวมข้อมูลส่วนบุคคลต้องเป็นไปตามกฎหมายคุ้มครองข้อมูลส่วนบุคคล (Personal Data Protection Act: PDPA)~\cite{pdpa} ซึ่งเป็นกฎหมายที่กำหนดวิธีการเก็บรวบรวม ใช้ และเปิดเผยข้อมูลส่วนบุคคลของบุคคลในประเทศไทย

สำหรับข้อมูลส่วนบุคลดังที่กล่าวมาข้างต้น ทางทีมผู้พัฒนาได้ทำการเข้ารหัสข้อมูล (encryption)~\cite{encryption} เพื่อเป็นการปกป้องข้อมูลส่วนบุคคลของผู้ใช้งาน และผู้ให้บริการเป็นที่เรียบร้อย
