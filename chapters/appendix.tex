\chapter{คู่มือการใช้งานระบบ Flowy}

\section{Login}
ขั้นตอนการเข้าสู่ระบบ ผู้ใช้งานต้องทำการกรอก username, password หลังจากนั้นกดปุ่ม \textbf{เข้าสู่ระบบ} หากผู้ใช้งานยังไม่มีบัญชีผู้ใช้งาน สามารถกด \textbf{สร้างบัญชีผู้ใช้} ได้

\section{Register}
ขั้นตอนการสร้างบัญชีผู้ใช้ ผู้ใช้งานต้องทำการกรอกข้อมูลดังต่อไปนี้
\begin{itemize}
    \item ชื่อ - นามสกุล
    \item อีเมล
    \item เบอร์โทรศัพท์
    \item รหัสผ่าน  
\end{itemize}
หากกรอกข้อมูลครบถ้วนกดปุ่มสมัครสมาชิก สำหรับผู้ใช้งานมีบัญชีอยู่แล้วสามารถกดปุ่ม \textbf{มีบัญชีอยู่แล้ว? เข้าสู่ระบบ} ได้เลย

\section{Account}
แสดงข้อมูลส่วนตัวของผู้ใช้งาน หากผู็ใช้งานต้องการทราบรายละเอียดการจองที่ผ่านมาสามารถกดได้ที่ \textbf{ประวัติการจอง} หากต้องการออกจากระบบผู้ใช้สามาถกดได้ที่ปุ่ม \textbf{ออกจากระบบ}

\section{Explore}
จะมีการแสดงผลในอยู่สองส่วนคือ สเปซที่ให้บริการ และโปรไฟล์
\begin{itemize}
    \item สเปซที่ให้บริการ ผู้ใช้บริการสามารถเลือกสเปซที่คุณต้องการได้จากการอ่านรายละเอียดที่แสดงผล โดยมีข้อมูลที่แสดงผลลดังต่อไปนี้
    \begin{itemize}
        \item ชื่อสเปซ
        \item เวลาเปิด - ปิดสเปซ
        \item ที่อยู่ของสเปซ
        \item ราคาค่าใช้งานเปซต่อชั่วโมง
        \item ความเฉพาะเจาะจง        
    \end{itemize}
    \item โปรไฟล์ ผู้ใช้บริการสามารถเข้ามาดูข้อมูลส่วนตัว และประวัติการจองสเปซได้จากส่วนนี้ 
\end{itemize}

\section{Place information}
จะแสดงข้อมูลรายละเอียดทั้งหมดของสเปซนั้น ๆ โดยจะแสดงข้อมูลดังต่อไปนี้
\begin{itemize}
    \item ชื่อสเปซ
    \item เวลาเปิด - ปิดสเปซ
    \item ที่อยู่ของสเปซ
    \item ราคาค่าใช้งานเปซต่อชั่วโมง
    \item สิ่งอำนวยความสะดวก
    \item ความเฉพาะเจาะจง
\end{itemize}
เมื่อผู้ใช้อ่านรายละเอียดแล้วต้องการจองสเปซ สามารถกดปุ่ม \textbf{จอง} ด้านล่างของหน้าจอเพื่อเข้าสู้ขั้นตอนของการจอง หากไม่ต้องกานจองสเปซนั้นแล้ว
ผู้ใช้งานสามารถกดปุ่ม \textbf{ย้อนกลับ} ได้ที่บนมุมขวาของหน้าจอ

\section{Booking  customer amount}
ผู้ใช้ต้องทำการระบบุจำนวนผู้เข้าใช้บริการสเปวนนั้น เมื่อผู้ใช้งานทำการเลือกจำนวนผู้ใช้งานเรียนร้อยแล้ว ผู้ใช้งานต้องกดปุ่ม \textbf{ถัดไป} เพื่อเข้าสู้หน้าเลือกรูปแบบโต๊ะ/ที่นั่ง
หากไม่ต้องการกลับสู่หน้าที่แล้วผู้ใช้งานสามารถกดปุ่ม \textbf{ย้อนกลับ} ได้ที่บนมุมขวาของหน้าจอ

\section{Booking desk}
ผู้ใช้ต้องทำการเลือกโต๊ะ/ที่นั่งที่แสดงผลได้ตามความต้องการ หลังจากนั้นกดปุ่ม \textbf{เลือก} ระบบจะพาไปสู่หน้าเลือกเวลาเข้าใช้บริการ
หากไม่ต้องการกลับสู่หน้าที่แล้วผู้ใช้งานสามารถกด \textbf{ย้อนกลับ} ได้ที่บนมุมขวาของหน้าจอ

\section{Booking time slot}
ผู้ใช้งานต้องทำการระบุเวลาที่ต้องการใช้งานโดยสามารถกดเลือกเวลาที่ว่าง เมื่อผู้ใช้กดเลือกเวลาที่ต้องการใช้งานสีที่แสดผลจะถูกเปลี่ยนเป็นสีแดง หากช่วงเวลาที่ไม่ว่างสล๊อตที่แสดงผลจะแสดงผลในสีเทาและไม่สามารถกดได้ \\
เมื่อผู้ใช้งานเลือกเวลาที่ต้องเข้างานเรียบร้อย กดปุ่มจองเลยตอนนี้ระบบจะพาไปสู่หน้าชำระค่าบริการ หากไม่ต้องการกลับสู่หน้าที่แล้วผู้ใช้งานสามารถกดปุ่มย้อนกลับได้ที่บนมุมขวาของหน้าจอ

\section{Payment}
สำหรับการจ่ายเงินจะมีส่วนแสดงผลอยู่สองส่วน คือ รายละเอียดค่าใช้บริการ และการชำระค่าบริการ
\begin{itemize}
    \item รายละเอียดค่าใช้บริการจะแสดงราคาที่ผู้ใช้บริการต้องททำการชำระโดยจะมีรายละเอียดดังนี้
    \begin{itemize}
        \item ค่าใช้บริการต่อชั่วโมง
        \item จำนวนผู้ใช้บริการสเปซ
        \item จำนวนชั่วโมงในการใช้บริการสเปซ        
    \end{itemize}
    \item การชำระค่าบริการ จะมีอยู่สองวิธี
    \begin{itemize}
        \item บัตรเครดิต/เดบิต
        \item พร้อมเพย์ QR        
    \end{itemize}
\end{itemize}
เมื่อชำระค่าบริการเรียบร้อยละบบจะพาไปสู่หน้าตั๋ว

\section{Ticket}
จะแสดงข้อมูลรายละเอียดการจองทั้งหมดโดยจะมีรายละเอียดที่แสดงผลดังนี้
\begin{itemize}
    \item ชื่อสเปซ และที่อยู่ของสเปซ
    \item วันที่จอง
    \item จำนวนชั่วโมงที่จอง
    \item QR code สำหรับการยืนยันข้อมูล
\end{itemize}
เมื่อผู้ใช้งานต้องการเดินทางไปที่สเปซสามารถกดปุ่ม \textbf{นำทางด้วย google maps} หรือถ้าผู้ใช้งานต้องการกลับสู่หน้าหลัก สามารถกดปุ่ม \textbf{กลับสู่หน้าหลัก} ได้

\chapter{คู่มือการใช้งานระบบ Flowider}

\section{Login}
ขั้นตอนการเข้าสู่ระบบ ผู้ใช้งานต้องทำการกรอก username, password หลังจากนั้นกดปุ่ม \textbf{เข้าสู่ระบบ} หากผู้ใช้งานยังไม่มีบัญชีผู้ใช้งาน สามารถกด \textbf{สร้างบัญชีผู้ใช้} ได้

\section{Register}
ขั้นตอนการสร้างบัญชีผู้ใช้ ผู้ใช้งานต้องทำการกรอกข้อมูลดังต่อไปนี้
\begin{itemize}
    \item ชื่อ - นามสกุล
    \item อีเมล
    \item เบอร์โทรศัพท์
    \item ชื่อธนาคาร
    \item หมายเลขบัญชีธนาคาร    
    \item รหัสผ่าน  
\end{itemize}
หากกรอกข้อมูลครบถ้วนกดปุ่มสมัครสมาชิก สำหรับผู้ใช้งานมีบัญชีอยู่แล้วสามารถกดปุ่ม \textbf{มีบัญชีอยู่แล้ว? เข้าสู่ระบบ} ได้เลย

\section{Dashboard}
หลังจากทำการเข้าสู้ระบบสำเร็จ จะหน้าแสดงลิตส์ของสเปซที่ผู้ให้บริการที่ได้ทำการลงทะเบียนไว้กับทาง flowy เมื่อกดเข้าสู่สเปซนั้น ๆ จะพบกับข้อมูลลการบุ๊กกิ้งของผู้ใช้บริการที่ได้ทำการจองเข้ามาในแต่ละวัน \\
หากผู้ให้บริการต้องการเพิ่มสเปซสามารถทำได้ด้วยการกดปุ่มจัดการที่แผงควบคุมทางด้านล่างตรงกลาง (รูปบ้าน) และถ้าผู้ใช้บริการต้องการดูข้อมูลส่วนตัวสามารถดูได้จากการกดปุ่มรูปตั้งค่า (รูปฟันเฟือง)

\section{Schedule}
หน้าแสดงบุ๊กกิ้งของสเปซนั้น ๆ ในแต่ละวัน หากต้องการย้อนกลับสู้หน้า dashboard สามารถทำได้โดยการกดปุ่ม \textbf{ย้อนกลับ} ที่บนมุมขวาของหน้าจอ

\section{Management}
หน้าจัดการสเปซ เป็นหน้าที่ผู้ให้บริการสามารถทำการเพิ่มสเปซ หรือทำการแก้ไขข้อมูลของสเปซได้ โดยการเพิ่มสเปซผู้ให้บริการสามารถทำได้โดยการกดปุ่ม \textbf{สร้างสเปซใหม่ +} และถ้าผู้ให้บริการต้องการแก้ไขข้อมูลของแต่ละสเปซที่ได้ทำการลงทะเบียนไว้ สามารถทำได้โดยการกดปุ่ม \textbf{แก้ไข} ที่มุมขวาล่างของแต่ละสเปซ \\
หากผู้ให้บริการต้องการลิตส์ของสเปซที่ได้ทำการลงทะเบียนสามารถกดที่แผงควบคุมทางด้านล่างด้านซ้าย (รูปปฏิทิน) และถ้าผู้ใช้บริการต้องการดูข้อมูลส่วนตัวสามารถดูได้จากการกดปุ่มรูปตั้งค่า (รูปฟันเฟือง)

\section{Add place}
เมื่อกดปุ่ม \textbf{สร้างสเปซใหม่ +} ระบบจะพาไปสู้หน้าเตรียมข้อมูลเพื่อใช้ในการลงทะเบียน มีขั้นตอนอยู่สองส่วนหลัก คือ ส่วนของการอธิบายเกี่ยวกับพื้นที่ของคุณ และส่วนของการเตรียมความพร้อมสำหรับโต๊ะแรก โดยมีข้อมูลที่ต้องเตรียมดังนี้
\begin{itemize}
    \item ข้อมูลเกี่ยวกับสเปซของคุณ
    \item ข้อมูลที่นั่งในสเปซของคุณ
\end{itemize}
หลังจากเตรียมข้อมูลเสร็จแล้วกดปุ่มกำเนินการต่อเพื่อทำการลงทะเบียนโดยทำการข้อมูลที่ได้บอกในขั้นตอนการเตรียมข้อมูลในข้างต้น เริ่มจาก ประเภทสเปซของคุณ, รายละเอียดของสเปซ, ความเฉพาะเจาะจง, สิ่งอำนวยความสะดวก, รูปภาพสำหรับสเปซของคุณ, ตำแหน่งสเปซของคุณ? และประเภทโต๊ะที่มีในสเปซ หลังจากกรอกข้อมูลครบถ้วนผู้ให้บริการสามารถกดปุ่มสร้างสเปซเพื่อทำการสร้างสเปซ

\section{Edit place}
หน้าจัดการของมูลของสปซที่คุณลงทะเบียนไว้ โดยมีข้อมูลที่สามารถแก้ไขได้ดังนี้
\begin{itemize}
    \item รูปภาพของสเปซ
    \item ข้อมูลทั่วไป
    \begin{itemize}
        \item ชื่อสเปซ
        \item ที่อยู่
        \item ประเภทสเปซ        
    \end{itemize}
    \item เวลาทำการ
    \begin{itemize}
        \item เวลาเปิด
        \item เวลาปิด        
    \end{itemize}
    \item ราคาต่อชั่วโมง (บาท)
    \item ความเฉพาะเจาะจง
    \begin{itemize}
        \item สูบบุหรี่ได้
        \item ใช้เสียงดังได้
        \item เงียบพิเศษ
        \item บรรยากาศดี
    \end{itemize}
    \item สิ่งอำนวยความสะดวก
    \begin{itemize}
        \item ปลั๊กไฟ/ปลั๊กพ่วง
        \item อินเทอร์เนต
        \item ห้องน้ำ
        \item จอแสดงผล
        \item สาย HDMI/สายสำหรับแสดงผล
        \item การดูแลอันอบอุ่นจากเจ้าของ
        \item เครื่องปรับอากาศ
        \item ขนม/เครื่องดื่ม
        \item การรักษาความปลอดภัย
    \end{itemize}
    \item ประเภทของโต๊ะที่มีในสเปซ
    \begin{itemize}
        \item ชื่อของโต๊ะ
        \item คำบรรยายโต๊ะ
        \item ประเภทของโต๊ะ
        \item จำนวนคนเข้าใชเบริการในโซนนี้
        \item รูปภาพของโต๊ะ        
    \end{itemize}
\end{itemize}
โดยการแก้ไขข้อมูลผู้ให้บริการสามารถกดที่คำว่า \textbf{แก้ไข} ที่มุมบนขวาของแต่ละส่วนเพื่อทำการแก้ไขข้อมูล เมื่อทำการการแก้ไขข้อมูลเสร็จสิ้นต้องทำการกดปุ่ม \textbf{บันทึก} เพื่อยืนยันการแก้ไข หากไม่ต้องการแก้ไชช้อมูลในส่วนนั้น ๆ ผู้ให้บริการสามารถกดที่ปุ่ม \textbf{ย้อนกลับ} ได้ที่มุมขวาบนของแต่ละส่วน \\
เมื่อต้องการกลับสู้หน้าจัดการสเปซผู้ใช้งานสามารถกดปุ่ม \textbf{ย้อนกลับ} ได้ที่มุมบนขวา

\section{Setting}
หน้าแสดงรายละเอียดของผู้ให้บริการ โดยจะมข้อมูลที่แสดงผลดังต่อไปนี้
\begin{itemize}
    \item ชื่อผู้ให้บริการ (ชื่อจริง - นามสกุล)
    \item วันที่สมัครใช้บริการ
    \item อีเมล
    \item เลขที่บัญชี
    \item ชื่อธนาคาร
\end{itemize}
การแก้ไขข้อมูลสามารถทำได้โดยยการกดปุ่ม \textbf{แก้ไข} ที่บนมุมขวาเพื่อทำการแก้ไขข้อมูล เมื่อทำการแก้ไขข้อมูลเรียบร้อยสามารถกดปุ่ม \textbf{ยืนยันการแก้ไขข้อมูล} เพื่อบันทึก \\
การออกจากระบบผู้ให้บริการสามารถทำได้โดยการกดปุ่ม \textbf{ออกจากระบบ} \\
หากผู้ให้บริการต้องการลิตส์ของสเปซที่ได้ทำการลงทะเบียนสามารถกดที่แผงควบคุมทางด้านล่างด้านซ้าย   (รูปปฏิทิน) และถ้าผู้ใช้บริการต้องการเพิ่มสเปซสามารถทำได้ด้วยการกดปุ่มจัดการที่แผงควบคุมทางด้านล่างตรงกลาง (รูปบ้าน)


% Text for a section in the first appendix goes here.

% test ทดสอบฟอนต์ serif ภาษาไทย

% \textsf{test ทดสอบฟอนต์ sans serif ภาษาไทย}

% \verb+test ทดสอบฟอนต์ teletype ภาษาไทย+

% \texttt{test ทดสอบฟอนต์ teletype ภาษาไทย}

% \textbf{ตัวหนา serif ภาษาไทย \textsf{sans serif ภาษาไทย} \texttt{teletype ภาษาไทย}}

% \textit{ตัวเอียง serif ภาษาไทย \textsf{sans serif ภาษาไทย} \texttt{teletype ภาษาไทย}}

% \textbf{\textit{ตัวหนาเอียง serif ภาษาไทย \textsf{sans serif ภาษาไทย} \texttt{teletype ภาษาไทย}}}

% \url{https://www.example.com/test_ทดสอบ_url}

% \chapter{\ifenglish Manual\else คู่มือการใช้งานระบบ\fi}

% Manual goes here.
