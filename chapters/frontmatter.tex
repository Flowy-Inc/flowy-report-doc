\maketitle
\makesignature

\ifproject
\begin{abstractTH}
\par การสอบกลางภาคและการสอบปลายภาค เป็นช่วงเวลาที่นักศึกษาและนักเรียนล้วนแล้วแต่ต้องเผชิญ และเตรียมตัวให้พร้อมสำหรับการสอบเหล่านั้น
แต่ที่น่าสนใจคือช่วงเวลาของเทศการสอบไม่ได้เริ่มต้นและสิ้นสุดตามปฏิทินการศึกษาแต่เพียงเท่านั้น หากแต่เริ่มต้นตั้งแต่ช่วงการสะสางงานก่อนสอบซึ่งอาจกินเวลาก่อนหน้าประมาณ 1-2 สัปดาห์
และแน่นอนว่าในงานเหล่านั้นจะต้องมีงานกลุ่มด้วยเช่นกัน ซึ่งนักศึกษาหลายๆ คนเลือกที่จะหาสถานที่ที่เหมาะสมกับการทำงานร่วมกันยาวไปจนถึงช่วงอ่านหนังสือเตรียมตัวสอบกับกลุ่มเพื่อนอีกด้วย

แต่ปรากฏว่าอุปทานไม่สอดรับกับอุปสงค์ กล่าวคือสถานที่นักศึกษามองหานั้นมีไม่เพียงพอต่อความต้องการ หรืออาจมีเพียงพอแต่ไม่อำนวยความสะดวกใด ๆ ให้กับผู้ใช้งาน
ด้วยความที่เราเป็นนักศึกษาที่เผชิญกับปัญหานี้โดยตรง เราจึงพยายามพิสูจน์ว่าเราไม่ได้เผชิญกับปัญหานี้แต่เพียงกลุ่มเดียวพร้อมกับเป็นผู้สร้างทางเลือกที่จะเข้ามาแก้ปัญหานี้ให้กับตัวเราเองรวมถึงผู้อื่นที่ประสบปัญหานี้ด้วยเช่นกัน

โครงงานนี้จึงมีจุดมุ่งหมายในการสร้างสิ่งอำนวยความสะดวกหรือทางเลือกใหม่ให้กับนักเรียน นักศึกษา หรือวัยทำงานที่ประสบปัญหาเรื่องการหาพื้นที่อ่านหนังสือ ประชุม ทำงานกลุ่ม และกิจกรรมอื่น ๆ ผ่านการ-
\enskip บูรณาการณ์ความรู้ทั้งในและนอกแขนงวิศวกรรมคอมพิวเตอร์ ออกมาในรูปแบบของแพลตฟอร์มที่ใช้งานได้ทุกอุปกรณ์เพื่อแก้ไขปัญหาปริมาณหรือลักษณะของที่นั่งไม่ตอบโจทย์ความต้องการให้ได้มากที่สุด
\end{abstractTH}

\begin{abstract}
\par The mid-term and final examinations are inevitable periods that every student has to face.
Interestingly, the exam period does not start on the precise start and end dates indicated on the academic calendar.
Rather, it usually starts 1-2 weeks before the actual exam period, which we can refer to as the "Assignment Clearance" period.

During this period, students are required to complete all outstanding assignments before the actual exam period begins.
As a result, students are more likely to search for suitable places to complete their work, and some of them continue to use these places for exam preparation.

However, due to the high demand and low supply of suitable study spaces, students often struggle to find a place that meets their needs.
This can be a major obstacle for students who require a quiet, distraction-free environment to concentrate and study effectively.

As students ourselves, we have experienced this problem firsthand and understand the difficulties that come with it.
Therefore, we initiated this project to integrate various fields of knowledge and provide an alternative solution that offers comfort and convenience for students facing similar struggles.

Our project aims to create a platform that offers a wide range of study spaces for students to choose from,
based on their specific needs and preferences. Through our platform, students can easily search for and book study spaces that are equipped with
essential amenities such as Wi-Fi, comfortable seating, and adequate lighting.

We believe that our project will not only benefit students but also contribute to the overall productivity and success of educational institutions.
By providing students with comfortable and convenient study spaces, we hope to alleviate some of the stress and anxiety associated with exam periods
and promote a more positive learning environment.

In summary, our project seeks to address the issue of inadequate study spaces during exam periods by providing a comprehensive platform
that connects students with suitable study spaces. We hope that our project will help students to perform better in their exams
and ultimately achieve their academic goals.
\end{abstract}

\iffalse
\begin{dedication}
This document is dedicated to all Chiang Mai University students.

Dedication page is optional.
\end{dedication}
\fi % \iffalse

\begin{acknowledgments}
โครงงานนี้จะไม่มีทางสำเร็จลุล่วงได้เลย ถ้าหากขาดความเอาใจใส่และการดูแลเป็นอย่างดีจาก อ.ดร.ชินวัตร อิศราดิสัยกุล ผู้เป็นอาจารย์ที่ปรึกษาโครงงานของเรา แม้ว่าช่วงการสอบกลางภาคของวิชา 261492
อาจเจอกับเหตุการณ์ที่ทำให้สูญเสียความมั่นใจไปเล็กน้อย แต่อาจารย์ชินวัตรก็ช่วยเรียกความมั่นใจของเรากลับมา อีกทั้งพวกเราต้องขอขอบพระคุณอาจารย์ภาสกร แช่มประเสริฐและอาจารย์พฤษภ์ บุญมา
ผู้เป็นกรรมการสอบตลอดห้วงเวลาของวิชา 261491 และ 261492 โดยให้คำแนะนำที่เป็นประโยชน์ให้กับเราเสมอมาทั้งที่เกี่ยวข้องกับการเรียนก็ดีหรือเกี่ยวข้องกับสภาวะความเป็นไปของโลกจริงก็ดี
และที่สำคัญที่สุดคือพี่ผู้เป็นผู้ดูแล Rimnim Hostel ที่คอยถามไถ่และให้กำลังใจพร้อมสนับสนุนโครงงานนี้ในอีกช่องทางหนึ่ง

และผมขอขอบคุณชนาธิป สงจันทึก (แบงค์) ในฐานะที่เป็นคู่หูผู้ร่วมทำโครงงานนี้มาด้วยกัน ที่ได้พิสูจน์ตัวเองและทำงานเข้าขากับเราเป็นอย่างดี เราประทับใจในความหัวไวและไฝ่รู้ของแบงค์มาก
ดีใจที่ได้ร่วมงานกันครั้งนี้
    
ท้ายที่สุด ผมขอเป็นตัวแทนของกลุ่มเรากล่าวขอบคุณทุกกำลังใจที่ไม่อาจกล่าวถึงได้หมดตลอดช่วงเวลาที่เราทุ่มเทแรงกายและใจให้กับโครงงานนี้เป็นอย่างสูง
\acksign{2023}{4}{5}
\end{acknowledgments}%
\fi % \ifproject

\contentspage

\ifproject
\figurelistpage

\tablelistpage
\fi % \ifproject

% \abbrlist % this page is optional

% \symlist % this page is optional

% \preface % this section is optional
